\documentclass{article}
\usepackage[utf8]{inputenc}
\usepackage[normalem]{ulem}

\title{Guide des commandes utiles pour LaTeX}
\author{Wolfyz}
\date{\today}

\begin{document}

\maketitle

\paragraph{Qu'est ce que le LaTex ? }

\section{Commandes de base}
\begin{itemize}
\item \textbackslash{}begin\{document\} : Commence le document.
\item \textbackslash{}end\{document\} : Termine le document.
\item \textbackslash{}title\{Titre\} : Définit le titre du document.
\item \textbackslash{}author\{Auteur\} : Définit l'auteur du document.
\item \textbackslash{}date\{Date\} : Définit la date du document.
Tips :  Vous pouvez mettre un \today pour mettre la date du jour 
\item \textbackslash{}maketitle : Affiche le titre du document.
\end{itemize}


\section{Commandes de mise en forme}
\begin{itemize}
\item \textbackslash{}textbf\{texte\} : Met le texte en \textbf{gras} .
\item \textbackslash{}textit\{texte\} : Met le texte en \textit{italique}.
\item \textbackslash{}underline\{texte\} : \underline{Souligne} le texte.
\item \textbackslash{}emph\{texte\} : Met \emph{en évidence le texte} (généralement en italique).
\item \textbackslash{}textsc\{texte\} : Met \textsc{le texte} en petites capitales.
\item \sout{Text to strike out}
\end{itemize}

\end{document}